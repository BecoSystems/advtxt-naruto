%%%%%%%%%%%%%%%%%%%%%%%%%%%%%%%%%%%%%%%%%%%%%%%%%%%%%%%%%%%%%%%%%%%%%%%%%%%%%%%%%%%%%%%%
% Criação de Fluxograma usando LaTeX
%
% Assunto: Aventura em que se o usuário escolher a opção errada ele morre
%
% Autores:
%     Filipe Lucca Alacoque Vaz
%
% Coordenação:
%     Prof. Dr. Ruben Carlo Benante
%
% Data: 2024-04-25
%%%%%%%%%%%%%%%%%%%%%%%%%%%%%%%%%%%%%%%%%%%%%%%%%%%%%%%%%%%%%%%%%%%%%%%%%%%%%%%%%%%%%%%%


%%%%%%%%%%%%%%%%%%%%%%%%%%%%%%%%%%%%%%%%%%%%%%%%%%%%%%%%%%%%%%%%%%%%%%%%%%%%%%%%%%%%%%%%
% Para gerar o PDF use o comando make com o makefile configurado:
%
%    $ make ext-programa2-benante-sobrenome1-sobrenome2.pdf
%
% O conteúdo do makefile é composto dos 3 seguintes comandos que ficam assim automatizados:
%    $ pdflatex ex11-fluxograma.tex -o ex11-fluxograma.pdf
%    $ bibtex biblio
%    $ pdflatex ex11-fluxograma.tex -o ex11-fluxograma.pdf


%%%%%%%%%%%%%%%%%%%%%%%%%%%%%%%%%%%%%%%%%%%%%%%%%%%%%%%%%%%%%%%%%%%%%%%%%%%%%%%%%%%%%%%%
% preambulo %%%%%%%%%%%%%%%%%%%%%%%%%%%%%%%%%%%%%%%%%%%%%%%%%%%%%%%%%%%%%%%%%%%%%%%%%%%%
\documentclass[a4paper,12pt]{article} %twocolumn
\usepackage[left=2.5cm,right=2cm,top=2.5cm,bottom=2cm]{geometry}
\usepackage[utf8]{inputenc} % letras acentuadas
\usepackage[portuguese]{babel} % tradução de títulos
\usepackage[colorlinks]{hyperref}
\usepackage{tikz} % para adicionar fluxogramas
\usepackage{algorithm} % ambiente para índice de algoritmos
\usepackage{algpseudocode} % fonte e estilo do algoritmo
\usepackage{graphicx} % permite adicionar imagens
\usepackage{indentfirst} % indenta o primeiro parágrafo também
\usepackage{url} % permite adicionar links de URLs e emails
% \usepackage{natbib}
%[noend]

\DeclareUrlCommand\email{\urlstyle{mm}} % comando para email bonito
\floatname{algorithm}{Algoritmo} % tradução da palavra algoritimo no ambiente de índice

\usetikzlibrary{shapes.geometric, shapes.symbols,arrows} % ajuste do tikz para incluir formas e setas

%%%%%%%%%%%%%%%%%%%%%%%%%%%%%%%%%%%%%%%%%%%%%%%%%%%%%%%%%%%%%%%%%%%%%%%%%%%%%%%%%%%%%%%%
% capa %%%%%%%%%%%%%%%%%%%%%%%%%%%%%%%%%%%%%%%%%%%%%%%%%%%%%%%%%%%%%%%%%%%%%%%%%%%%%%%%%
\title{Fluxograma: Aventura Naruto}
\author{Filipe Lucca Alacoque Vaz}

\begin{document}

\maketitle

%%%%%%%%%%%%%%%%%%%%%%%%%%%%%%%%%%%%%%%%%%%%%%%%%%%%%%%%%%%%%%%%%%%%%%%%%%%%%%%%%%%%%%%%
% definicao dos blocos do fluxograma (tikz) %%%%%%%%%%%%%%%%%%%%%%%%%%%%%%%%%%%%%%%%%%%%

\tikzstyle{line} = [draw, -latex']
\tikzstyle{startend} = [draw, ellipse,fill=red!20, minimum height=2em, node distance=1.55cm]
\tikzstyle{print} = [tape, fill=blue!20, draw, draw=black, minimum width=3cm, minimum height=1.4cm, text width=4.5em, text centered, tape bend top=none, tape bend height=0.2cm, node distance=1.55cm]
\tikzstyle{input} = [trapezium, trapezium left angle=60, trapezium right angle=90, minimum width=3cm, minimum height=1cm, text centered, draw=black, fill=blue!30, node distance=1.95cm]
\tikzstyle{process} = [rectangle, minimum width=3cm, minimum height=1cm, text centered, draw=black, fill=orange!30, node distance=1.55cm]

\tikzstyle{block} = [rectangle, draw, fill=blue!20, text width=5em, text centered, rounded corners, minimum height=4em, node distance=1.55cm]
\tikzstyle{decisionb} = [diamond, draw, fill=blue!20, text width=4.5em, text badly centered, inner sep=0pt, node distance=1.55cm]
\tikzstyle{decision} = [diamond, minimum width=3cm, minimum height=1cm, text centered, draw=black, fill=green!30, node distance=2.25cm]
\tikzstyle{empty} = [circle, fill=white, minimum width=0.01mm, node distance=2.55cm]

%%%%%%%%%%%%%%%%%%%%%%%%%%%%%%%%%%%%%%%%%%%%%%%%%%%%%%%%%%%%%%%%%%%%%%%%%%%%%%%%%%%%%%%%
% resumo %%%%%%%%%%%%%%%%%%%%%%%%%%%%%%%%%%%%%%%%%%%%%%%%%%%%%%%%%%%%%%%%%%%%%%%%%%%%%%%

\begin{abstract}

\textbf{Assunto:} Programa tal tal

% descrever em poucas palavras seu projeto aqui

O programa tal e tal faz isso e isso. Neste artigo iremos apresentar o seu fluxograma completo
% e (opcionalmente) o seu algoritmo.

Após a modelagem do fluxograma e desenvolvimento da lógica de programação em algoritmo,
o programa será implementado na Linguagem de Programação \texttt{C}


\textbf{Local:} Escola Politécnica de Pernambuco - UPE/POLI

\textbf{Órgão Financiador:} N/A

\textbf{Caracterização:} Modelagem, Projeto e Implementação de Software em Linguagem \texttt{C}

% Este é o fim do resumo.

\end{abstract}


%%%%%%%%%%%%%%%%%%%%%%%%%%%%%%%%%%%%%%%%%%%%%%%%%%%%%%%%%%%%%%%%%%%%%%%%%%%%%%%%%%%%%%%%
% artigo %%%%%%%%%%%%%%%%%%%%%%%%%%%%%%%%%%%%%%%%%%%%%%%%%%%%%%%%%%%%%%%%%%%%%%%%%%%%%%%
% seção de introdução %%%%%%%%%%%%%%%%%%%%%%%%%%%%%%%%%%%%%%%%%%%%%%%%%%%%%%%%%%%%%%%%%%
\section{Introdução}

% Descrever melhor seu projeto aqui

Este programa faz isso, isso e \texttt{também em fonte mono-espaçada faz isso}.

O programa será modelado em \textit{fluxograma} em uma primeira fase, em seguida
sua lógica será desenvolvida em formato de \textit{algoritmo}, para então
na terceira fase ser implementado em Linguagem de Programação \texttt{C}.

Exemplo de letra em \textit{itálico é com textit}, e em \textbf{bold face é com textbf} e \texttt{tal tal tal é mono-espaçada type-writer}.

%%%%%%%%%%%%%%%%%%%%%%%%%%%%%%%%%%%%%%%%%%%%%%%%%%%%%%%%%%%%%%%%%%%%%%%%%%%%%%%%%%%%%%%%
% seção de objetivos %%%%%%%%%%%%%%%%%%%%%%%%%%%%%%%%%%%%%%%%%%%%%%%%%%%%%%%%%%%%%%%%%%%
\section{Fluxograma}

% adicionar aqui o fluxograma

\begin{tikzpicture}

    % colocar nodos

    % Nós
    \node (start) [startend] {Início};
    \node (intro) [print, below of=start] {Você é o Naruto. Você está enfrentando um oponente forte.};
    \node (objeto) [print, below of=intro] {Escolha uma arma};
    \node (escolha1) [input, below of=objeto] {Espada, Samehada ou Foice?};
    \node (verbo) [print, below of=escolha1] {Escolha a ação};
    \node (escolha2) [input, below of=verbo] {Atacar, Defender ou Correr?};
    \node (valido) [decision, below of=escolha2] {Entrada válida?};

    % Se Não no válido
    \node (retornaObjeto) [print, left of=valido, xshift=-4cm] {Escolha novamente uma arma};

    % Se Sim no válido
    \node (espadaAtacar) [decision, below of=valido] {Espada + Atacar?};
    \node (morreu1) [print, right of=espadaAtacar, xshift=4cm] {Morreu};

    \node (espadaDefender) [decision, below of=espadaAtacar] {Espada + Defender?};
    \node (morreu2) [print, right of=espadaDefender, xshift=4cm] {Morreu};

    \node (espadaCorrer) [decision, below of=espadaDefender] {Espada + Correr?};
    \node (morreu3) [print, right of=espadaCorrer, xshift=4cm] {Morreu};

    \node (samehadaAtacar) [decision, below of=espadaCorrer] {Samehada + Atacar?};
    \node (morreu4) [print, right of=samehadaAtacar, xshift=4cm] {Morreu};

    \node (samehadaDefender) [decision, below of=samehadaAtacar] {Samehada + Defender?};
    \node (venceu) [print, right of=samehadaDefender, xshift=4cm] {Venceu};

    \node (samehadaCorrer) [decision, below of=samehadaDefender] {Samehada + Correr?};
    \node (morreu5) [print, right of=samehadaCorrer, xshift=4cm] {Morreu};

    \node (foiceAtacar) [decision, below of=samehadaCorrer] {Foice + Atacar?};
    \node (morreu6) [print, right of=foiceAtacar, xshift=4cm] {Morreu};

    \node (foiceDefender) [decision, below of=foiceAtacar] {Foice + Defender?};
    \node (morreu7) [print, right of=foiceDefender, xshift=4cm] {Morreu};

    \node (foiceCorrer) [decision, below of=foiceDefender] {Foice + Correr?};
    \node (morreu8) [print, right of=foiceCorrer, xshift=4cm] {Morreu};

    \node (end) [startend, right of=venceu, xshift=4cm] {Fim};

    % \node (vazio1) [empty, right of=fim, node distance=4cm] {};
    % Desenhar as setas
    \path [line] (start) -- (intro);
    \path [line] (intro) -- (objeto);
    \path [line] (objeto) -- (escolha1);
    \path [line] (escolha1) -- (verbo);
    \path [line] (verbo) -- (escolha2);
    \path [line] (escolha2) -- (valido);

    \path [line] (valido.west) -- node[anchor=south] {Não} (retornaObjeto.east);
    \path [line] (valido) -- node[anchor=west] {Sim} (espadaAtacar);

    \path [line] (espadaAtacar.east) -- node[anchor=south] {Sim} (morreu1) -- (end);
    \path [line] (espadaAtacar) -- node[anchor=west] {Não} (espadaDefender);

    \path [line] (espadaDefender.east) -- node[anchor=south] {Sim} (morreu2) -- (end);
    \path [line] (espadaDefender) -- node[anchor=west] {Não} (espadaCorrer);

    \path [line] (espadaCorrer.east) -- node[anchor=south] {Sim} (morreu3) -- (end);
    \path [line] (espadaCorrer) -- node[anchor=west] {Não} (samehadaAtacar);

    \path [line] (samehadaAtacar.east) -- node[anchor=south] {Sim} (morreu4) -- (end);
    \path [line] (samehadaAtacar) -- node[anchor=west] {Não} (samehadaDefender);

    \path [line] (samehadaDefender.east) -- node[anchor=south] {Sim} (venceu) -- (end);
    \path [line] (samehadaDefender) -- node[anchor=west] {Não} (samehadaCorrer);

    \path [line] (samehadaCorrer.east) -- node[anchor=south] {Sim} (morreu5) -- (end);
    \path [line] (samehadaCorrer) -- node[anchor=west] {Não} (foiceAtacar);

    \path [line] (foiceAtacar.east) -- node[anchor=south] {Sim} (morreu6) -- (end);
    \path [line] (foiceAtacar) -- node[anchor=west] {Não} (foiceDefender);

    \path [line] (foiceDefender.east) -- node[anchor=south] {Sim} (morreu7) -- (end);
    \path [line] (foiceDefender) -- node[anchor=west] {Não} (foiceCorrer);

    \path [line] (foiceCorrer.east) -- node[anchor=south] {Sim} (morreu8) -- (end);

\end{tikzpicture}

\clearpage % inicia próxima seção em nova página
%%%%%%%%%%%%%%%%%%%%%%%%%%%%%%%%%%%%%%%%%%%%%%%%%%%%%%%%%%%%%%%%%%%%%%%%%%%%%%%%%%%%%%%%
% seção de justificativa %%%%%%%%%%%%%%%%%%%%%%%%%%%%%%%%%%%%%%%%%%%%%%%%%%%%%%%%%%%%%%%
% \section{Algoritmo}

% adicionar aqui o algoritmo (opcional)

% \clearpage % inicia próxima seção em nova página
%%%%%%%%%%%%%%%%%%%%%%%%%%%%%%%%%%%%%%%%%%%%%%%%%%%%%%%%%%%%%%%%%%%%%%%%%%%%%%%%%%%%%%%%
% Autores %%%%%%%%%%%%%%%%%%%%%%%%%%%%%%%%%%%%%%%%%%%%%%%%%%%%%%%%%%%%%%%%%%%%%%%%%%%%%%
\section*{Detalhamento dos Autores}

%%%%%%%%%%%%%%%%%%%%%%%%%%%%%%%%%%%%%%%%%%%%%%%%%%%%%%%%%%%%%%%%%%%%%%%%%%%%%%%%%%%%%%%%
% Discentes %%%%%%%%%%%%%%%%%%%%%%%%%%%%%%%%%%%%%%%%%%%%%%%%%%%%%%%%%%%%%%%%%%%%%%%%%%%%
\subsection*{Discentes}

\begin{enumerate}
    \item \textbf{Nome Completo:} Fulano de Tal Um
    \begin{description}
        \item [Email:] \email{blabla@poli.br}
        \item [Endereço:]
        \item [Matrícula:]
        \item [CPF:]
        \item [RG:]
        \item [Telefone:]
        \item [Currículo Lattes:] \url{http://lattes.cnpq.br/nnnnn}
    \end{description}

    \item \textbf{Nome Completo:} Fulano de Qual Dois
    \begin{description}
        \item [Email:] \email{blabla@poli.br}
        \item [Endereço:]
        \item [Matrícula:]
        \item [CPF:]
        \item [RG:]
        \item [Telefone:]
        \item [Currículo Lattes:] \url{http://lattes.cnpq.br/nnnnn}
    \end{description}

    \item \textbf{Nome Completo:} Fulano de Como Três
    \begin{description}
        \item [Email:] \email{blabla@poli.br}
        \item [Endereço:]
        \item [Matrícula:]
        \item [CPF:]
        \item [RG:]
        \item [Telefone:]
        \item [Currículo Lattes:] \url{http://lattes.cnpq.br/nnnnn}
    \end{description}

    % \item \textbf{Nome Completo:} Fulano de tal
    % \begin{description}
        % \item [Email:] \email{blabla@poli.br}
        % \item [Endereço:]
        % \item [Matrícula:]
        % \item [CPF:]
        % \item [RG:]
        % \item [Telefone:]
        % \item [Currículo Lattes:] \url{http://lattes.cnpq.br/nnnnn}
    % \end{description}

%     \item \textbf{Nome Completo:} Fulano de tal
%     \begin{description}
%         \item [Email:] \email{blabla@poli.br}
%         \item [Endereço:]
%         \item [Matrícula:]
%         \item [CPF:]
%         \item [RG:]
%         \item [Telefone:]
%         \item [Currículo Lattes:] \url{http://lattes.cnpq.br/nnnnn}
%     \end{description}
\end{enumerate}


%%%%%%%%%%%%%%%%%%%%%%%%%%%%%%%%%%%%%%%%%%%%%%%%%%%%%%%%%%%%%%%%%%%%%%%%%%%%%%%%%%%%%%%%
% Docentes %%%%%%%%%%%%%%%%%%%%%%%%%%%%%%%%%%%%%%%%%%%%%%%%%%%%%%%%%%%%%%%%%%%%%%%%%%%%%
\subsection*{Docentes}

\begin{enumerate}
    \item \textbf{Nome Completo:} Ruben Carlo Benante
    \begin{description}
        \item [Email:] \email{rcb@upe.br}
        \item [Matrícula:] 11238-0
        \item [Currículo Lattes:] \url{http://lattes.cnpq.br/3366717378277623}
    \end{description}
\end{enumerate}


%%%%%%%%%%%%%%%%%%%%%%%%%%%%%%%%%%%%%%%%%%%%%%%%%%%%%%%%%%%%%%%%%%%%%%%%%%%%%%%%%%%%%%%%
% referências bibliográficas %%%%%%%%%%%%%%%%%%%%%%%%%%%%%%%%%%%%%%%%%%%%%%%%%%%%%%%%%%%
%\section*{Referências Bibliográficas}

% cite todos, mesmo os não referenciados %%%%%%%%%%%%%%%%%%%%%%%%%%%%%%%%%%%%%%%%%%%%%%%
\nocite{*}


%%%%%%%%%%%%%%%%%%%%%%%%%%%%%%%%%%%%%%%%%%%%%%%%%%%%%%%%%%%%%%%%%%%%%%%%%%%%%%%%%%%%%%%%
% se necessario %%%%%%%%%%%%%%%%%%%%%%%%%%%%%%%%%%%%%%%%%%%%%%%%%%%%%%
% troca autor and autor por autor & autor, na bibliografia. O dcu usa "and"
%\renewcommand{\harvardand}{\&} % troca and pro &. O dcu usa "and"

% Estilos de bibliografia %%%%%%%%%%%%%%%%%%%%%%%%%%%%%%%%%%%%%%%%%%%%%%%%%%%%%%%%%%%%%%
% \bibliographystyle{abnt-alf} % Estilo alfabético da ABNT. Opção [num] para estilo numérico
% \bibliographystyle{apalike}
% \bibliographystyle{dcu} %citacao como (autor and autor, ano). Parece apalike. Rev. Control. Automacao. Use com harvard
% \bibliographystyle{agsm} % padrao harvard fica (autor & autor ano).
\bibliographystyle{acm}

%%%%%%%%%%%%%%%%%%%%%%%%%%%%%%%%%%%%%%%%%%%%%%%%%%%%%%%%%%%%%%%%%%%%%%%%%%%%%%%%%%%%%%%%
% arquivo de banco de dados das referências %%%%%%%%%%%%%%%%%%%%%%%%%%%%%%%%%%%%%%%%%%%%
% renomear para o número do exercício correto
% o arquivo de bibliografia pode se chamar qualquer coisa, isso não muda o comando de gerar o PDF.
% Por exemplo para 'mybiblio.bib', use \bibliography{mybiblio} e os comandos pdflatex e bibtex continuam os mesmos identicos com exN.
\bibliography{biblio}

\end{document}
